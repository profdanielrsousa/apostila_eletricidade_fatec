% Título do capítulo
\capitulo{Lei de Ohm e Corrente Elétrica}\label{cap2}

\mais{

\begin{center}
    \Large \textbf{Objetivo}
\end{center}

Ao final deste capítulo, espera-se que o estudante seja capaz de:

\begin{itemize}
    \item Compreender o conceito de corrente elétrica e sua definição matemática.
    \item Diferenciar corrente real e corrente convencional.
    \item Reconhecer os diversos tipos de materiais condutores: metálicos, líquidos e gasosos.
    \item Utilizar corretamente a Primeira Lei de Ohm para relacionar tensão, corrente e resistência.
    \item Aplicar a Segunda Lei de Ohm para calcular a resistência em função das propriedades do condutor.
    \item Interpretar tabelas de resistividade e compreender a influência da temperatura na resistência elétrica.
    \item Efetuar cálculos com múltiplos e submúltiplos das unidades elétricas.
    \item Resolver problemas práticos envolvendo a Lei de Ohm em diferentes contextos.
\end{itemize}
}

\secao{Introdução}
\index{seção}

Neste capítulo, abordaremos a Lei de Ohm, uma das mais importantes leis da eletricidade, que estabelece a relação entre tensão elétrica, corrente e resistência. Vamos também relembrar conceitos fundamentais da eletrodinâmica, explorar os tipos de corrente e discutir os materiais condutores.

\section{Eletrodinâmica: O Início}

A eletrodinâmica teve grande avanço com a invenção da pilha elétrica por Alessandro Volta. Antes disso, os estudos em eletricidade limitavam-se a fenômenos eletrostáticos, como a eletrização por atrito.

A pilha de Volta consistia em discos alternados de cobre e zinco, separados por feltros embebidos em solução ácida. Essa configuração permitia gerar uma diferença de potencial entre os terminais.

\section{Geradores de Corrente}

Os geradores podem ser classificados em:

\begin{itemize}
    \item \textbf{Geradores de corrente contínua (CC):} como pilhas e baterias.
    \item \textbf{Geradores de corrente alternada (CA):} usados em usinas hidroelétricas, termelétricas, entre outras.
\end{itemize}

\section{Sentido da Corrente Elétrica}

A corrente elétrica é o movimento ordenado dos elétrons através de um condutor. Esse movimento real ocorre do polo negativo para o polo positivo do gerador. No entanto, por convenção, a corrente elétrica é representada como fluindo do polo positivo para o negativo, chamada de \textbf{corrente convencional}.

Ambos os sentidos não ocorrem simultaneamente. A corrente real representa o fluxo físico de elétrons; a convencional é adotada por convenção para facilitar a análise de circuitos.

\section{Materiais Condutores de Eletricidade}

Existem diferentes tipos de materiais capazes de conduzir corrente elétrica:

\begin{itemize}
    \item \textbf{Condutores metálicos:} metais como cobre, alumínio, ouro, prata e ferro, com presença de elétrons livres.
    \item \textbf{Condutores líquidos:} soluções eletrolíticas, como água com sais ou ácidos, que possuem íons livres.
    \item \textbf{Condutores gasosos:} requerem alta diferença de potencial para se ionizarem e permitirem a passagem de corrente.
\end{itemize}

\section{Definição de Corrente Elétrica}

A corrente elétrica ($I$) é definida como a quantidade de carga ($Q$), em coulombs, que passa por uma seção transversal de um condutor em um intervalo de tempo ($\Delta t$):

\[
I = \frac{Q}{\Delta t}
\]

A unidade de medida da corrente elétrica é o \textbf{ampère (A)}. Um ampère equivale a um coulomb por segundo:

\[
1\,\text{A} = 1\,\text{C/s}
\]

\section{Medição de Corrente e Tensão Elétrica}

Para medir corrente elétrica, utilizamos um \textbf{amperímetro}, que deve ser conectado em \textbf{série} com a carga no circuito.

Para medir tensão elétrica (diferença de potencial), usamos um \textbf{voltímetro}, que deve ser conectado em \textbf{paralelo} com o componente.

Hoje em dia, o \textbf{multímetro} é um instrumento muito utilizado para medir várias grandezas, como corrente, tensão e resistência.


\section{Resistência Elétrica}

A resistência elétrica ($R$) representa a oposição que um material oferece à passagem da corrente elétrica. Depende do tipo de material, do comprimento e da área da seção transversal do condutor.

A unidade de medida da resistência é o \textbf{ohm ($\Omega$)}:

\[
1\,\Omega = \frac{1\,\text{V}}{1\,\text{A}}
\]

\section{Primeira Lei de Ohm}

Formulada por Georg Simon Ohm, esta lei estabelece que, em um condutor ôhmico, a corrente elétrica ($I$) é diretamente proporcional à tensão elétrica ($V$) e inversamente proporcional à resistência ($R$):

\[
I = \frac{V}{R}
\]

As formas alternativas da equação são:

\[
V = R \cdot I \quad \text{ou} \quad R = \frac{V}{I}
\]

\textbf{Triângulo REI:} Um método mnemônico útil para lembrar as fórmulas:

\[
\begin{array}{c}
\boxed{V} \\
\boxed{R \quad I}
\end{array}
\]

\section{Segunda Lei de Ohm}

A Segunda Lei de Ohm descreve como a resistência elétrica de um fio depende de suas propriedades físicas:

\[
R = \rho \cdot \frac{L}{A}
\]

Onde:

\begin{itemize}
    \item $R$: resistência elétrica (\(\Omega\))
    \item $\rho$: resistividade do material (\(\Omega \cdot \text{m}\) ou \(\Omega \cdot \text{mm}^2/\text{m}\))
    \item $L$: comprimento do condutor (m)
    \item $A$: área da seção transversal (\(\text{mm}^2\) ou \(\text{mm}^2)\))
\end{itemize}

\subsection{Tabela de Resistividade de Materiais}

\begin{center}
\begin{tabular}{|l|c|}
\hline
\textbf{Material} & \textbf{Resistividade (\(\Omega \cdot \text{mm}^2/\text{m}\)) a 20$^{\circ}$C} \\
\hline
Prata & \(1,59 \times 10^{-2}\) \\
Cobre & \(1,68 \times 10^{-2}\) \\
Ouro & \(2,44 \times 10^{-2}\) \\
Alumínio & \(2,82 \times 10^{-2}\) \\
Constantan (liga) & \(49 \times 10^{-2}\) \\
Níquel & \(6,99 \times 10^{-2}\) \\
\hline
\end{tabular}
\end{center}

\textbf{Observações:}
\begin{itemize}
    \item Materiais com menor resistividade conduzem melhor.
    \item Materiais como o ouro, embora mais caros, são usados em contatos elétricos por sua resistência à oxidação.
    \item Ligas como \textit{Constantan} são preferidas na fabricação de resistores devido à baixa variação com a temperatura.
\end{itemize}

\subsection{Influência da Temperatura}

A resistividade varia com a temperatura. Em metais condutores, a resistividade aumenta com a elevação da temperatura. Já em ligas específicas (como Constantan), essa variação é muito menor, o que as torna ideais para componentes eletrônicos.

\section{Exercícios Resolvidos}

\subsection*{Exemplo 1: O Pássaro no Fio}
Um pássaro pousa em um fio de alta tensão que conduz uma corrente de $1000\,\text{A}$. A resistência por metro do fio é $5 \times 10^{-5}\,\Omega/\text{m}$ e a distância entre os pés do pássaro é de $6\,\text{cm}$. Qual a diferença de potencial entre os pés do pássaro?

\textbf{Resolução:}
\begin{itemize}
    \item Convertendo $6\,\text{cm} = 0{,}06\,\text{m}$
    \item Resistência correspondente:
    \[
    R = \left(5 \times 10^{-5}\right) \times 0{,}06 = 3 \times 10^{-6}\,\Omega
    \]
    \item Tensão elétrica:
    \[
    V = R \cdot I = \left(3 \times 10^{-6}\right) \cdot 1000 = 3 \times 10^{-3}\,\text{V} = 3\,\text{mV}
    \]
\end{itemize}

\textbf{Conclusão:} A diferença de potencial é muito pequena para causar dano ao pássaro: apenas $3\,\text{mV}$.

\subsection*{Exemplo 2: Resistência de um Cabo}
Um cabo de transmissão é formado por 7 fios de cobre, cada um com $10\,\text{mm}^2$ de área. Qual a resistência elétrica para um comprimento de $1\,\text{km}$, dado que $\rho = 2{,}1 \times 10^{-2}\,\Omega \cdot \text{mm}^2/\text{m}$?

\textbf{Resolução:}
\begin{itemize}
    \item Área total: $A = 7 \times 10 = 70\,\text{mm}^2$
    \item Comprimento: $L = 1000\,\text{m}$
    \item Aplicando a segunda lei de Ohm:
    \[
    R = \rho \cdot \frac{L}{A} = 2{,}1 \times 10^{-2} \cdot \frac{1000}{70} = 0{,}3\,\Omega
    \]
\end{itemize}

\subsection*{Exemplo 3: Tensão Elétrica}
Um resistor de $100\,\Omega$ é percorrido por uma corrente de $2{,}5\,\text{A}$. Qual a tensão?

\[
V = R \cdot I = 100 \cdot 2{,}5 = 250\,\text{V}
\]

\subsection*{Exemplo 4: Resistência}
Um resistor é ligado a $110\,\text{V}$ e conduz uma corrente de $5\,\text{A}$. Qual sua resistência?

\[
R = \frac{V}{I} = \frac{110}{5} = 22\,\Omega
\]

\subsection*{Exemplo 5: Corrente Elétrica}
Um resistor de $200\,\Omega$ está ligado a uma fonte de $300\,\text{V}$. Qual a corrente?

\[
I = \frac{V}{R} = \frac{300}{200} = 1{,}5\,\text{A}
\]

\subsection*{Exemplo 6: Corrente em Circuito Automotivo}
Um resistor de $6\,\Omega$ é ligado a uma bateria de $12\,\text{V}$. Qual a corrente elétrica?

\[
I = \frac{12}{6} = 2\,\text{A}
\]

\section{Notação Científica e Unidades}

\begin{itemize}
    \item \textbf{Múltiplos:} 
        \begin{itemize}
            \item $1\,\text{k} = 10^3 = 1000$
            \item $1\,\text{M} = 10^6 = 1\,000\,000$
        \end{itemize}
    \item \textbf{Submúltiplos:}
        \begin{itemize}
            \item $1\,\text{m} = 10^{-3}$
            \item $1\,\mu = 10^{-6}$
        \end{itemize}
    \item \textbf{Representação correta:} número + espaço + unidade (ex.: $2{,}5\,\text{A}$).
    \item \textbf{Separador decimal:} utilizar vírgula (ex.: $3{,}14$), e não ponto.
    \item \textbf{Números grandes:} não utilizar ponto entre milhares (ex.: $10000$, e não $10.000$).
\end{itemize}

O material extra sobre notação científica e unidades serão disponibilizados no material complementar.

\section{Considerações Finais}

Neste capítulo, abordamos a corrente elétrica, os tipos de condutores, a resistência e as leis fundamentais de Ohm. Compreender esses conceitos é essencial para o estudo de circuitos elétricos e aplicações em eletrônica, instalações e equipamentos.

\section{Material Complementar}

\midia{O site do INMETRO possui um documento sobre o Sistema Internacional de Medidas, e em especial ao \textbf{capítulo 3} (Múltiplos e submúltiplos decimais das unidades do SI), \textbf{seção 5.4.2} (Símbolos das grandezas e das unidades), \textbf{seção 5.4.3} (Escrita do valor de uma grandeza) e \textbf{seção 5.4.4} (Escrita dos números e separador decimal), são importantes para a disciplina. O material está disponível no link
\url{https://www.gov.br/inmetro/pt-br/centrais-de-conteudo/publicacoes/documentos-tecnicos-em-metrologia/si_versao_final.pdf}.}

\noindent\textbf{Próximo capítulo:} \hyperref[cap3]{Associação de Resistores.}
