\noindent \textsc{Curso Superior em Gestão da Tecnologia da Informação - Instituto Federal Baiano, Campus Bom Jesus da Lapa}\\

\noindent 
Caro estudante,

Bem-vindos ao Ciclo (19 a 30 de abril) da disciplina de Banco de Dados I que tratará os conceitos teóricos para aprofundar seus conhecimentos sobre Banco de Dados no curso Superior em Gestão da Tecnologia da Informação do Instituto Federal Baiano – Campus Bom Jesus da Lapa.

Para que seu estudo se torne proveitoso e prazeroso, este ciclo foi organizada em 2 semanas, com temas e subtemas que, por sua vez, são subdivididos em seções (tópicos), atendendo aos objetivos do processo de ensino-aprendizagem.

Nesta primeira semana, trataremos sobre os conceitos de modelos de dados, procuraremos compreender os conceitos gerais sobre modelagem de dados (modelo de dados, Entidade-Relacionamento, Relacionamentos, Atributos Chaves, Normalização e Padronização).  

%No capítulo 2, descreveremos [...]. No capítulo 3, detalharemos [...]. Finalmente, no capítulo 4 refletiremos um pouco sobre [...].

Esperamos que, até o final deste ciclo vocês possam:

- Ampliar a compreensão sobre Banco de Dados;
- Conceituar os Sistemas Gerenciadores de Banco de Dados (SGDB);
- Compreender a importância da normalização de padronização;

Para tanto, teremos vídeo aulas expositivas que serão disponibilizadas em nosso Ambiente Virtual de Aprendizagem (AVA) do Instituto Federal Baiano, campus Bom Jesus da Lapa.

Porém, antes de iniciar a leitura, gostaríamos que vocês parassem um instante para refletir sobre algumas questões: Qual é a importância de Banco de Dados para as organizações?

Não se preocupe.  Não queremos que vocês respondam de imediato essa questão.  Mas esperamos que, até o final, vocês tenham respostas e também formulem outras perguntas.

Vamos, então, iniciar nossas aulas? Bons estudos!

