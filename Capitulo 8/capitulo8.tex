% Título do capítulo
\capitulo{Aplicações das Leis de Kirchhoff}\label{cap8}

\mais{

\begin{center}
    \Large \textbf{Objetivo}
\end{center}

Ao final deste capítulo, espera-se que o estudante seja capaz de:

\begin{itemize}
    \item Aplicar de forma prática as Leis de Kirchhoff na resolução de circuitos elétricos com múltiplas malhas e fontes.
    \item Montar as equações de nós e malhas a partir de circuitos esquemáticos.
    \item Escolher de forma estratégica os sentidos das correntes e trabalhar com sinais adequados.
    \item Resolver sistemas de equações simultâneas por meio do método da substituição.
    \item Interpretar corretamente o sinal das correntes obtidas para avaliar seus sentidos reais.
    \item Verificar a coerência física dos resultados obtidos, como tensões, quedas e sentidos de corrente.
\end{itemize}
}

%\secao{Geradores e Receptores Elétricos}
%\index{seção}

\section{Introdução}
Neste capítulo, retomamos as Leis de Kirchhoff para resolver circuitos mais complexos, compostos por duas malhas e múltiplos resistores. Utilizaremos o método da substituição para resolver os sistemas de equações gerados pelas Leis dos Nós e das Malhas.

\section{Exercício 1 – Duas Malhas com Fontes Diferentes}

\subsection{Descrição do Circuito}
O circuito contém:

\begin{itemize}
    \item Fonte de $20\,\text{V}$ na malha 1
    \item Fonte de $40\,\text{V}$ na malha 2
    \item Resistores de $5\,\Omega$, $10\,\Omega$ e $5\,\Omega$
    \item Correntes: $I_1$, $I_2$ e $I_3$, uma para cada ramo
\end{itemize}

\textbf{Observação:} O resistor de $10\,\Omega$ está compartilhado entre as duas malhas.

\subsection{1ª Equação – Lei dos Nós (1ª Lei de Kirchhoff)}
Todas as três correntes entram no nó analisado:

\[
I_1 + I_2 + I_3 = 0 \tag{1}
\]

\subsection{2ª Equação – Malha 1 (2ª Lei de Kirchhoff)}
Percurso no sentido da corrente $I_1$ (horário):

\[
20 - 5I_1 + 10I_2 = 0 \tag{2}
\]

\subsection{3ª Equação – Malha 2}
Percurso no sentido da corrente $I_3$ (antihorário):

\[
40 + 10I_2 - 5I_3 = 0 \tag{3}
\]

\subsection{Simplificação das Equações}
Dividindo todas as equações por $5$:

\[
\text{Equação 2:} \quad 4 - I_1 + 2I_2 = 0 \tag{2'}
\]
\[
\text{Equação 3:} \quad 8 + 2I_2 - I_3 = 0 \tag{3'}
\]

Essas são as três equações base que usaremos para resolver o sistema via substituição.

\subsection{Resolução do Sistema (Exercício 1)}

\subsubsection{Passo 1 – Isolar $I_1$ na Equação (2')}
\[
4 - I_1 + 2I_2 = 0 \Rightarrow I_1 = 4 + 2I_2 \tag{4}
\]

\subsubsection{Passo 2 – Isolar $I_3$ na Equação (3')}
\[
8 + 2I_2 - I_3 = 0 \Rightarrow I_3 = 8 + 2I_2 \tag{5}
\]

\subsubsection{Passo 3 – Substituir (4) e (5) na Equação dos Nós (1)}

\[
I_1 + I_2 + I_3 = 0
\Rightarrow (4 + 2I_2) + I_2 + (8 + 2I_2) = 0
\]
\[
\Rightarrow 12 + 5I_2 = 0 \Rightarrow I_2 = -\frac{12}{5} = -2{,}4\,\text{A}
\]

\subsubsection{Passo 4 – Substituir $I_2$ para encontrar $I_1$ e $I_3$}

\[
I_1 = 4 + 2(-2{,}4) = 4 - 4{,}8 = -0{,}8\,\text{A}
\]
\[
I_3 = 8 + 2(-2{,}4) = 8 - 4{,}8 = 3{,}2\,\text{A}
\]

\subsection{Interpretação dos Resultados}

\begin{itemize}
    \item $I_2 = -2{,}4\,\text{A}$ → sentido real oposto ao atribuído inicialmente.
    \item $I_1 = -0{,}8\,\text{A}$ → também sentido oposto ao escolhido.
    \item $I_3 = 3{,}2\,\text{A}$ → mesmo sentido da atribuição inicial.
\end{itemize}

\textbf{Conclusão:} Correntes negativas indicam que o sentido real de circulação é contrário ao que foi considerado na análise. O valor absoluto permanece válido.

\subsection{Verificação – Quedas de Tensão}

\begin{itemize}
    \item Queda no resistor de $5\,\Omega$ (malha 1): $V = R \cdot I = 5 \cdot 0{,}8 = 4\,\text{V}$
    \item Queda no resistor de $10\,\Omega$ (em comum): $10 \cdot 2{,}4 = 24\,\text{V}$
    \item Queda no resistor de $5\,\Omega$ (malha 2): $5 \cdot 3{,}2 = 16\,\text{V}$
\end{itemize}

\textbf{Fonte da malha 1:}
\[
20\,\text{V} = 4 + 24 \Rightarrow \text{ok}
\]

\textbf{Fonte da malha 2:}
\[
40\,\text{V} = 24 + 16 \Rightarrow \text{ok}
\]

As quedas de tensão estão coerentes com os valores de corrente encontrados.

\section{Exercício 2 – Duas Fontes e Três Malhas com Correntes Interligadas}

\subsection{Descrição do Circuito}
O circuito contém:

\begin{itemize}
    \item Fonte de $50\,\text{V}$ (malha 1)
    \item Fonte de $20\,\text{V}$ (malha 2)
    \item Resistores: $R_1 = 5\,\Omega$, $R_2 = 10\,\Omega$, $R_3 = 5\,\Omega$
    \item Correntes: $I_1$, $I_2$ e $I_3$
\end{itemize}

\textbf{Topologia:} o resistor de $10\,\Omega$ é comum às duas malhas, e os outros dois resistores completam os laços. Cada malha possui uma fonte e uma parte da resistência total.

\subsection{Equação da Malha 1}
\[
50 - 5I_1 + 10I_3 = 0 \tag{1}
\]

\subsection{Equação da Malha 2}
\[
20 - 5I_2 - 10I_3 = 0 \tag{2}
\]

\subsection{Lei dos Nós}
\[
I_1 + I_2 + I_3 = 0 \tag{3}
\]

\subsection{Dividindo todas as equações por 5}
\[
10 - I_1 + 2I_3 = 0 \Rightarrow I_1 = 10 + 2I_3 \tag{1'}
\]
\[
4 - I_2 - 2I_3 = 0 \Rightarrow I_2 = 4 - 2I_3 \tag{2'}
\]

\subsection{Resolução do Sistema (Exercício 2)}

\subsubsection{Passo 1 – Substituir $I_1$ e $I_2$ na Equação dos Nós (3)}

\[
I_1 + I_2 + I_3 = 0
\Rightarrow (10 + 2I_3) + (4 - 2I_3) + I_3 = 0
\]
\[
\Rightarrow 14 + I_3 = 0 \Rightarrow I_3 = -14\,\text{A}
\]

\subsubsection{Passo 2 – Substituir $I_3$ para encontrar $I_1$ e $I_2$}

\[
I_1 = 10 + 2(-14) = 10 - 28 = -18\,\text{A}
\]
\[
I_2 = 4 - 2(-14) = 4 + 28 = 32\,\text{A}
\]

\subsection{Interpretação dos Resultados}

\begin{itemize}
    \item $I_3 = -14\,\text{A}$: corrente oposta ao sentido atribuído.
    \item $I_1 = -18\,\text{A}$: também oposta.
    \item $I_2 = 32\,\text{A}$: mesmo sentido atribuído.
\end{itemize}

\textbf{Conclusão:} Sinais negativos indicam que o sentido real das correntes é o contrário ao definido inicialmente. O valor numérico das correntes permanece válido.

\subsection{Verificação – Quedas de Tensão}

\begin{itemize}
    \item Queda no resistor $R_1 = 5\,\Omega$ com $I_1 = 18\,\text{A}$:
    \[
    V = 5 \cdot 18 = 90\,\text{V}
    \]
    \item Queda no resistor $R_2 = 10\,\Omega$ com $I_3 = 14\,\text{A}$:
    \[
    V = 10 \cdot 14 = 140\,\text{V}
    \]
    \item Queda no resistor $R_3 = 5\,\Omega$ com $I_2 = 32\,\text{A}$:
    \[
    V = 5 \cdot 32 = 160\,\text{V}
    \]
\end{itemize}

\textbf{Observação:} As fontes fornecem menos energia do que a soma das quedas, indicando que o circuito exige ajuste de polaridades ou sentido de malha, reforçando a importância da verificação final com os sinais corretos.

\section{Considerações Finais}

Neste capítulo, você aprendeu a:

\begin{itemize}
    \item Aplicar as Leis de Kirchhoff em circuitos com múltiplas malhas e fontes.
    \item Utilizar o método da substituição para resolver sistemas de equações.
    \item Analisar o sinal das correntes obtidas e interpretar seus sentidos reais.
    \item Validar os resultados por meio das quedas de tensão nos resistores.
\end{itemize}

\vspace{0.5cm}
\noindent\textbf{Próximo capítulo:} Potência elétrica em corrente contínua e consumo de energia em circuitos.