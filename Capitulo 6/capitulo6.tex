% Título do capítulo
\capitulo{Geradores e Receptores Elétricos}\label{cap6}

\mais{

\begin{center}
    \Large \textbf{Objetivo}
\end{center}

Ao final deste capítulo, espera-se que o estudante seja capaz de:

\begin{itemize}
    \item Compreender o funcionamento de geradores elétricos e os diferentes tipos de fontes de energia.
    \item Distinguir entre geradores ideais e reais, e aplicar o modelo equivalente com resistência interna.
    \item Calcular potência útil, potência dissipada e potência total fornecida por um gerador.
    \item Determinar o rendimento de um gerador e compreender os fatores que influenciam sua eficiência.
    \item Identificar a condição de máxima transferência de potência e suas implicações práticas.
    \item Compreender o funcionamento dos receptores elétricos e o conceito de força contra eletromotriz.
    \item Aplicar equações para receptores e calcular as potências associadas ao seu funcionamento.
    \item Analisar a associação de geradores em série e paralelo, identificando vantagens e cuidados.
    \item Reconhecer os efeitos da corrente circulante e como evitá-la em associações mal planejadas.
\end{itemize}
}

%\secao{Geradores e Receptores Elétricos}
%\index{seção}

\section{Introdução}
Neste capítulo, abordaremos os principais tipos de \textbf{geradores e receptores elétricos}, analisando seus princípios de funcionamento, modelos ideais e reais, rendimento, potência útil e dissipação, além das associações em série e paralelo.

\section{O Que São Geradores Elétricos?}
Geradores elétricos são dispositivos capazes de \textbf{converter algum tipo de energia em energia elétrica}. As fontes podem ser:

\begin{itemize}
    \item \textbf{Eletroquímicas} (pilhas, baterias);
    \item \textbf{Eletromagnéticas} (usinas hidroelétricas, termelétricas);
    \item \textbf{Termoelétricas} (junção de dois metais distintos);
    \item \textbf{Pisoelétricas} (pressão em cristais);
    \item \textbf{Fotoelétricas} (luz incidindo sobre materiais semicondutores).
\end{itemize}

\subsection{Exemplos de Geração de Energia}
\begin{itemize}
    \item \textbf{Pilhas e Baterias:} reações químicas geram tensão elétrica.
    \item \textbf{Usinas Hidrelétricas:} energia potencial da água move turbinas que ativam geradores.
    \item \textbf{Efeito Termoelétrico:} emenda de dois metais diferentes, como cobre e alumínio, gera pequena tensão.
    \item \textbf{Cristais Pisoelétricos:} pressão sobre o cristal gera tensão (ex: isqueiros, nebulizadores).
    \item \textbf{Células Fotovoltaicas:} fótons solares liberam elétrons, gerando corrente elétrica.
\end{itemize}

\subsection*{Exemplo Prático: Usina Henry Borden}
Localizada em Cubatão (SP), a usina recebe água da Represa Billings que desce por tubulações com 700 metros de desnível. A energia potencial da água é transformada em energia elétrica por meio de turbinas conectadas a geradores do tipo Pelton.

\subsection{Geradores de Tensão e Corrente}
Modelamos os geradores como fontes ideais ou reais:

\begin{itemize}
    \item \textbf{Gerador de Tensão Ideal:} mantém a mesma tensão independentemente da carga.
    \item \textbf{Gerador de Corrente Ideal:} mantém corrente constante mesmo com variações na resistência da carga.
\end{itemize}

\subsubsection{Na Prática:}
Todo gerador possui uma \textbf{resistência interna ($r$)} que causa perdas, diminuindo a tensão efetiva entregue à carga conforme a corrente aumenta.

\subsection{Modelo Real de Gerador de Tensão Contínua}
Representamos um gerador real por uma fonte de força eletromotriz ($\mathcal{E}$) em série com uma resistência interna ($r$).

\[
U = \mathcal{E} - r \cdot I
\]

Onde:
\begin{itemize}
    \item $U$: tensão nos terminais da fonte com carga conectada (V)
    \item $\mathcal{E}$: tensão em vazio (sem carga)
    \item $r$: resistência interna do gerador ($\Omega$)
    \item $I$: corrente fornecida à carga (A)
\end{itemize}

\subsubsection{Curva Característica do Gerador}
Com dois pontos experimentais (tensão em aberto e corrente de curto-circuito), pode-se traçar a \textbf{reta de carga}, cuja inclinação corresponde à resistência interna $r$.

\[
\text{Ponto 1: } I = 0 \Rightarrow U = \mathcal{E} \quad \text{(sem carga)}\\
\text{Ponto 2: } U = 0 \Rightarrow I = \frac{\mathcal{E}}{r} \quad \text{(curto-circuito)}
\]

\subsection{Potência em Geradores Reais}
A energia fornecida por um gerador se divide entre:

\begin{itemize}
    \item \textbf{Potência útil ($P_u$):} entregue à carga.
    \item \textbf{Potência dissipada ($P_d$):} perdida na resistência interna do gerador.
    \item \textbf{Potência total ($P_t$):} fornecida pela fonte.
\end{itemize}

\subsubsection{Equações Fundamentais}

\[
P_t = \mathcal{E} \cdot I
\quad ; \quad
P_u = U \cdot I = (\mathcal{E} - r \cdot I) \cdot I
\quad ; \quad
P_d = r \cdot I^2
\]

\subsection{Rendimento do Gerador}
O rendimento ($\eta$) de um gerador é a razão entre a potência útil e a potência total fornecida:

\[
\eta = \frac{P_u}{P_t} = \frac{U \cdot I}{\mathcal{E} \cdot I} = \frac{U}{\mathcal{E}}
\quad \Rightarrow \quad
\eta = \frac{\mathcal{E} - r \cdot I}{\mathcal{E}}
\]

\textbf{Nota:} O rendimento pode ser expresso em forma decimal ou porcentagem.

\subsection{Condição de Máxima Transferência de Potência}
O gerador fornece \textbf{potência máxima à carga} quando:

\[
R_L = r
\]

Ou seja, a resistência da carga deve ser igual à resistência interna do gerador. Nesse caso:

\begin{itemize}
    \item A potência útil é máxima.
    \item O rendimento é de apenas 50\% (a outra metade é dissipada).
\end{itemize}

\[
P_{u_{\text{máx}}} = \frac{\mathcal{E}^2}{4r}
\]

\textbf{Importante:} essa condição é útil em sistemas que priorizam potência (como amplificadores), mas não é eficiente em sistemas que priorizam economia de energia (como fontes de alimentação).

\subsection{Exemplo Resolvido}
Um gerador real fornece uma tensão de $10\,\text{V}$ em vazio. Quando conectado a uma carga de $R_L = 4\,\Omega$, a tensão medida nos terminais cai para $8\,\text{V}$.

\textbf{a) Qual a resistência interna do gerador?}

\begin{itemize}
    \item Corrente na carga:
    \[
    I = \frac{U}{R_L} = \frac{8}{4} = 2\,\text{A}
    \]
    \item Resistência interna:
    \[
    \mathcal{E} = U + r \cdot I \Rightarrow 10 = 8 + 2r \Rightarrow r = 1\,\Omega
    \]
\end{itemize}

\textbf{b) Qual a potência útil, dissipada e total?}

\[
P_u = U \cdot I = 8 \cdot 2 = 16\,\text{W}
\quad ; \quad
P_d = r \cdot I^2 = 1 \cdot 2^2 = 4\,\text{W}
\quad ; \quad
P_t = \mathcal{E} \cdot I = 10 \cdot 2 = 20\,\text{W}
\]

\textbf{c) Qual o rendimento?}

\[
\eta = \frac{P_u}{P_t} = \frac{16}{20} = 0{,}8 = 80\%
\]

\section{Receptores Elétricos}
Receptores são dispositivos que consomem energia elétrica para transformá-la em outro tipo de energia (térmica, mecânica, luminosa etc.). Exemplos:

\begin{itemize}
    \item Motores elétricos
    \item Lâmpadas
    \item Aparelhos eletrodomésticos
    \item Dispositivos eletrônicos em geral
\end{itemize}

\subsection{Modelo Real do Receptor}
Um receptor real possui:
\begin{itemize}
    \item Uma resistência interna ($r$), que dissipa parte da energia.
    \item Uma tensão de contraposição ($\mathcal{E}_r$), que representa a energia útil convertida.
\end{itemize}

\textbf{Equação:}

\[
U = \mathcal{E}_r + r \cdot I
\]

\begin{itemize}
    \item $U$: tensão nos terminais do receptor (V)
    \item $\mathcal{E}_r$: contraposição eletromotriz (V)
    \item $r$: resistência interna do receptor ($\Omega$)
    \item $I$: corrente elétrica (A)
\end{itemize}

\subsection{Potência nos Receptores}
A potência absorvida total é:

\[
P_t = U \cdot I
\]

Ela se divide em:
\begin{itemize}
    \item \textbf{Potência útil:} $\mathcal{E}_r \cdot I$ (convertida em trabalho útil)
    \item \textbf{Potência dissipada:} $r \cdot I^2$ (perda em forma de calor)
\end{itemize}

\subsection{Exemplo Resolvido}
Um motor elétrico funciona com $220\,\text{V}$ e consome $2\,\text{A}$. Sabendo que sua resistência interna é de $5\,\Omega$, determine:

\begin{itemize}
    \item \textbf{a) A força contra eletromotriz ($\mathcal{E}_r$):}
    \[
    U = \mathcal{E}_r + r \cdot I \Rightarrow 220 = \mathcal{E}_r + 5 \cdot 2
    \Rightarrow \mathcal{E}_r = 210\,\text{V}
    \]

    \item \textbf{b) A potência total, útil e dissipada:}
    \[
    P_t = 220 \cdot 2 = 440\,\text{W}
    \quad ; \quad
    P_u = 210 \cdot 2 = 420\,\text{W}
    \quad ; \quad
    P_d = 5 \cdot 2^2 = 20\,\text{W}
    \]
\end{itemize}

\subsection{Observação Importante:}
Em geradores, a energia flui do gerador para a carga. Em receptores, a energia elétrica entra no componente e é parcialmente convertida e parcialmente dissipada.

\subsection{Receptor em Corrente Alternada (CA)}
Para motores e outros dispositivos em corrente alternada, conceitos como potência ativa, reativa e aparente devem ser considerados, mas isso será aprofundado em capítulos futuros.

\subsection{Associação de Geradores}
Geradores podem ser associados para fornecer maior tensão, corrente ou garantir redundância no sistema.

\subsubsection{a) Associação em Série}
\begin{itemize}
    \item As tensões se somam:
    \[
    \mathcal{E}_{eq} = \mathcal{E}_1 + \mathcal{E}_2 + \dots + \mathcal{E}_n
    \]
    \item As resistências internas também se somam:
    \[
    r_{eq} = r_1 + r_2 + \dots + r_n
    \]
    \item A corrente no circuito é a mesma para todos os geradores.
\end{itemize}

\textbf{Aplicações:} baterias de lanternas, controle de tensão em painéis solares.

\subsubsection{b) Associação em Paralelo}
\begin{itemize}
    \item Todos os geradores devem ter \textbf{a mesma tensão} ($\mathcal{E}$).
    \item As resistências internas combinam-se como resistores em paralelo.
    \item As correntes se dividem entre os geradores.
\end{itemize}

\[
\frac{1}{r_{eq}} = \frac{1}{r_1} + \frac{1}{r_2} + \dots + \frac{1}{r_n}
\]

\textbf{Atenção:} se as tensões forem diferentes, haverá \textbf{corrente circulante}, que pode causar sobreaquecimento e danificar os componentes.

\subsubsection{Exemplo: Corrente Circulante}
Dois geradores com $\mathcal{E}_1 = 12\,\text{V}$ e $\mathcal{E}_2 = 9\,\text{V}$ conectados em paralelo tendem a equilibrar tensões, e o gerador mais forte (12 V) tentará "carregar" o outro, criando uma corrente interna indesejada.

\textbf{Solução:} Sempre utilizar diodos ou circuitos de proteção quando houver risco de tensões diferentes em paralelo.

\subsection{Considerações Finais}
Neste capítulo, você aprendeu:

\begin{itemize}
    \item A diferença entre geradores ideais e reais.
    \item A influência da resistência interna na tensão fornecida.
    \item Como calcular a potência útil, dissipada e total.
    \item Como determinar o rendimento de um gerador.
    \item A condição de máxima transferência de potência.
    \item O funcionamento de receptores reais, como motores.
    \item Como associar geradores em série e paralelo de forma segura.
\end{itemize}

\section{Material Complementar}

\midia{

\begin{itemize}
    \item Matéria do reporter Goulart de Andrade em 2003 na Usina Henry Borden: \url{https://www.youtube.com/watch?v=jPjhsKvv8fM}.
    \item Canal Mundo da Elétrica, Usina Rio das Pedras: \url{https://www.youtube.com/watch?v=kjmFwB71tkk}

\end{itemize}

}


\vspace{0.5cm}
\noindent\textbf{Próximo capítulo:} \hyperref[cap7]{Leis de Kirchhoff.}


