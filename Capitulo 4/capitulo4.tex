% Título do capítulo
\capitulo{Exercícios sobre Associação de Resistores}\label{cap4}

\mais{

\begin{center}
    \Large \textbf{Objetivo}
\end{center}

Ao final deste capítulo, espera-se que o estudante seja capaz de:

\begin{itemize}
    \item Resolver exercícios aplicados sobre associação de resistores em série e paralelo.
    \item Aplicar corretamente fórmulas para cálculo da resistência equivalente.
    \item Realizar conversão entre unidades de resistência (ohm, quilohm, megaohm).
    \item Identificar visualmente associações em série, paralelo e mista, independentemente da orientação do desenho.
    \item Utilizar estratégias passo a passo para resolver associações mistas.
    \item Verificar a coerência dos resultados com base em princípios teóricos.
    \item Reforçar os conceitos estudados nos capítulos anteriores por meio da prática.
\end{itemize}
}

\secao{Exercícios sobre Associação de Resistores}
\index{seção}

\subsection{Revisão: Associação em Série}
Na associação em série, a \textbf{resistência equivalente} é a soma direta dos resistores ligados em sequência:

\[
R_{eq} = R_1 + R_2 + R_3 + \dots + R_n
\]

\textbf{Importante:} Todos os resistores devem estar na mesma unidade para que a soma seja válida.

\subsubsection{Exemplo 1}
Dois resistores de $100\,\Omega$ em série entre os pontos A e B:

\[
R_{eq} = 100 + 100 = 200\,\Omega
\]

\subsubsection{Exemplo 2}
Um resistor de $50\,\Omega$ e outro de $145\,\Omega$:

\[
R_{eq} = 50 + 145 = 195\,\Omega
\]

\subsubsection{Exemplo 3: Conversão de Unidades}
Um resistor de $1\,\text{k}\Omega$ e outro de $220\,\Omega$:

\[
1\,\text{k}\Omega = 1000\,\Omega
\quad \Rightarrow \quad
R_{eq} = 1000 + 220 = 1220\,\Omega
\]

\subsubsection{Exemplo 4}
$2{,}5\,\text{k}\Omega$ e $560\,\Omega$:

\[
2{,}5\,\text{k}\Omega = 2500\,\Omega
\quad \Rightarrow \quad
R_{eq} = 2500 + 560 = 3060\,\Omega
\]

\subsubsection{Exemplo 5}
Resistores: $100\,\text{k}\Omega$, $550\,\Omega$ e $6000\,\Omega$:

\[
100\,\text{k}\Omega = 100000\,\Omega
\quad \Rightarrow \quad
R_{eq} = 100000 + 550 + 6000 = 106550\,\Omega
\]

\subsubsection{Exemplo 6: Megaohms e kilohms}
Resistores: $1\,\text{M}\Omega$, $620\,\text{k}\Omega$, $1000\,\Omega$:

\[
1\,\text{M}\Omega = 1000000\,\Omega,\quad
620\,\text{k}\Omega = 620000\,\Omega
\quad \Rightarrow \quad
R_{eq} = 1000000 + 620000 + 1000 = 1621000\,\Omega
\]

\subsubsection{Nota:}
A ordem e disposição dos resistores (horizontal, vertical, zigue-zague) não altera a configuração elétrica. O importante é observar como os terminais estão conectados em sequência.

\subsubsection{Exemplo 7}
$72\,\text{k}\Omega$ + $330\,\text{k}\Omega$:

\[
R_{eq} = 72 + 330 = 402\,\text{k}\Omega
\]

Ou, em ohms:

\[
72000 + 330000 = 402000\,\Omega
\]

\subsubsection{Exemplo 8: Resistores em zigue-zague}
$30\,\text{k}\Omega$, $20\,\text{k}\Omega$, $10\,\text{k}\Omega$:

\[
R_{eq} = 30 + 20 + 10 = 60\,\text{k}\Omega = 60000\,\Omega
\]

\subsection{Revisão: Associação em Paralelo}
Na associação em paralelo, a \textbf{tensão elétrica é a mesma} em todos os resistores, e a \textbf{corrente se divide} de acordo com o valor da resistência.

A resistência equivalente é dada pela fórmula geral:

\[
\frac{1}{R_{eq}} = \frac{1}{R_1} + \frac{1}{R_2} + \dots + \frac{1}{R_n}
\]

\textbf{Importante:} O resultado da resistência equivalente será sempre \textbf{menor que o menor resistor da associação}.

\subsubsection{Casos Particulares}
\begin{itemize}
    \item \textbf{Resistores iguais:}
    \[
    R_{eq} = \frac{R}{n}
    \]

    \item \textbf{Dois resistores apenas:}
    \[
    R_{eq} = \frac{R_1 \cdot R_2}{R_1 + R_2}
    \]
\end{itemize}

\subsubsection{Exemplo 1: Dois resistores diferentes}
$R_1 = 100\,\Omega$, $R_2 = 50\,\Omega$

\[
\frac{1}{R_{eq}} = \frac{1}{100} + \frac{1}{50} = \frac{3}{100}
\Rightarrow
R_{eq} = \frac{100}{3} \approx 33{,}33\,\Omega
\]

\textit{Verificação:} $33{,}33\,\Omega < 50\,\Omega$ ✅

\subsubsection{Exemplo 2: Mesmo problema usando produto pela soma}
\[
R_{eq} = \frac{100 \cdot 50}{100 + 50} = \frac{5000}{150} = 33{,}33\,\Omega
\]

\subsubsection{Exemplo 3: Resistores de 25\,\Omega e 30\,\Omega}
\[
\frac{1}{R_{eq}} = \frac{1}{25} + \frac{1}{30}
= \frac{11}{150}
\Rightarrow R_{eq} \approx 13{,}63\,\Omega
\]

\textit{Verificação:} $13{,}63\,\Omega < 25\,\Omega$

\subsubsection{Exemplo 4: Três resistores distintos}
$R_1 = 30\,\Omega$, $R_2 = 60\,\Omega$, $R_3 = 90\,\Omega$

\[
\frac{1}{R_{eq}} = \frac{1}{30} + \frac{1}{60} + \frac{1}{90}
= \frac{6}{45} = \frac{2}{15}
\Rightarrow R_{eq} = \frac{15}{2} = 7{,}5\,\Omega
\]

\textit{Verificação:} $R_{eq} < 30\,\Omega$

\subsubsection{Exemplo 5: Três resistores iguais}
Três resistores de $30\,\Omega$:

\[
R_{eq} = \frac{30}{3} = 10\,\Omega
\]

Ou usando a fórmula geral:

\[
\frac{1}{R_{eq}} = \frac{1}{30} + \frac{1}{30} + \frac{1}{30}
= \frac{3}{30} = \frac{1}{10}
\Rightarrow R_{eq} = 10\,\Omega
\]

\subsubsection{Exemplo 6: Diversos valores}
$R_1 = 10\,\text{k}\Omega$, $R_2 = 300\,\Omega$, $R_3 = 1\,\text{k}\Omega$, $R_4 = 500\,\Omega$

\[
\frac{1}{R_{eq}} = \frac{1}{10000} + \frac{1}{300} + \frac{1}{1000} + \frac{1}{500}
\Rightarrow R_{eq} \approx 154{,}4\,\Omega
\]

\textit{Verificação:} $154{,}4\,\Omega < 300\,\Omega$ ✅

\subsection{Associação Mista de Resistores}
Associação mista ocorre quando resistores estão conectados em partes em série e outras em paralelo no mesmo circuito.

\subsubsection{Estratégia de Resolução}
\begin{enumerate}
    \item Identificar trechos que estão em paralelo e calcular sua resistência equivalente.
    \item Substituir os trechos calculados e verificar se restaram resistores em série.
    \item Repetir o processo até reduzir o circuito a uma única resistência equivalente entre os pontos A e B.
\end{enumerate}

\subsubsection{Exemplo 1: Paralelo seguido de série}
Resistores:
\[
\text{Paralelo: } R_1 = 100\,\Omega, R_2 = 200\,\Omega \quad\text{em paralelo}
\quad\text{em série com } R_3 = 300\,\Omega
\]

\begin{itemize}
    \item Primeiro, calcule o paralelo:
    \[
    \frac{1}{R_{12}} = \frac{1}{100} + \frac{1}{200} = \frac{3}{200} \Rightarrow R_{12} = \frac{200}{3} \approx 66{,}67\,\Omega
    \]
    \item Em seguida, calcule a série:
    \[
    R_{eq} = R_{12} + R_3 = 66{,}67 + 300 = 366{,}67\,\Omega
    \]
\end{itemize}

\subsubsection{Exemplo 2: Série seguida de paralelo}
Resistores:
\[
\text{Série: } R_1 = 300\,\Omega,\ R_2 = 400\,\Omega
\quad\text{em paralelo com } R_3 = 500\,\Omega
\]

\begin{itemize}
    \item Série:
    \[
    R_{12} = 300 + 400 = 700\,\Omega
    \]
    \item Paralelo com $R_3$:
    \[
    \frac{1}{R_{eq}} = \frac{1}{700} + \frac{1}{500} \Rightarrow R_{eq} \approx 291{,}67\,\Omega
    \]
\end{itemize}

\subsubsection{Exemplo 3: Dois paralelos seguidos de série}
\[
\text{Primeiro paralelo: } 1\,\text{k}\Omega\ \| \ 20\,\text{k}\Omega
\quad\text{Segundo paralelo: } 10\,\text{k}\Omega\ \| \ 2\,\text{k}\Omega
\]

\begin{itemize}
    \item Primeiro:
    \[
    \frac{1}{R_1} = \frac{1}{1000} + \frac{1}{20000} \Rightarrow R_1 \approx 952{,}38\,\Omega
    \]
    \item Segundo:
    \[
    \frac{1}{R_2} = \frac{1}{10000} + \frac{1}{2000} \Rightarrow R_2 \approx 1666{,}67\,\Omega
    \]
    \item Série final:
    \[
    R_{eq} = R_1 + R_2 = 952{,}38 + 1666{,}67 = 2619{,}05\,\Omega
    \]
\end{itemize}

\subsection{Dicas Finais e Encerramento}
\begin{itemize}
    \item Sempre converta todas as resistências para a mesma unidade antes de somar.
    \item Em série: soma direta das resistências.
    \item Em paralelo: o resultado é sempre menor que o menor resistor.
    \item Identifique visualmente os caminhos e não se prenda à forma do desenho (horizontal, vertical, zigue-zague).
    \item Use a fórmula do produto pela soma apenas para dois resistores.
    \item Para mistos, resolva por etapas: paralelo → substitui → soma em série.
\end{itemize}

\vspace{0.5cm}
\noindent\textbf{Próximo capítulo:} \hyperref[cap5]{Verificação Experimental da Lei de Ohm.}