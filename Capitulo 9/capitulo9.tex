% Título do capítulo
\capitulo{Instrumentação Elétrica e Eletrônica}\label{cap9}

\mais{

\begin{center}
    \Large \textbf{Objetivo}
\end{center}

Ao final deste capítulo, espera-se que o estudante seja capaz de:

\begin{itemize}
  \item Compreender os conceitos fundamentais de medição e instrumentação.
  \item Identificar os principais tipos de erros em medições elétricas.
  \item Distinguir os conceitos de exatidão e precisão.
  \item Conhecer os componentes e o funcionamento dos instrumentos analógicos e digitais.
  \item Classificar os instrumentos segundo suas aplicações e características construtivas.
\end{itemize}
}

%\secao{Geradores e Receptores Elétricos}
%\index{seção}

\section{Conceito de Medição}
Medir é estabelecer uma relação entre uma grandeza e outra da mesma espécie, geralmente comparando com um padrão. Em medidas elétricas, como tensão, corrente, resistência e potência, utilizamos instrumentos específicos, pois tais grandezas não são tangíveis.

O \textbf{padrão de medição} deve ter:
\begin{itemize}
  \item \textbf{Estabilidade:} não pode se alterar com o tempo ou com variações ambientais.
  \item \textbf{Reprodutibilidade:} deve ser possível reproduzir fielmente seus valores.
\end{itemize}

\section{Tipos de Erros de Medição}
\begin{enumerate}
  \item \textbf{Erros grosseiros:} causados por falha humana, como leitura errada, má ligação ou paralaxe.
  \item \textbf{Erros sistemáticos:} oriundos do instrumento ou método de medição.
    \begin{itemize}
      \item Instrumentais: desgaste, mau contato, oxidação.
      \item Ambientais: temperatura, umidade, pressão inadequadas.
    \end{itemize}
  \item \textbf{Erros aleatórios:} imprevisíveis, como ruídos elétricos, interferência de rádio, etc.
\end{enumerate}

\section{Exatidão e Precisão}
Apesar de usadas como sinônimos, possuem significados distintos:
\begin{itemize}
  \item \textbf{Exatidão:} proximidade do valor medido em relação ao valor real.
  \item \textbf{Precisão:} repetibilidade dos valores medidos.
\end{itemize}

\section{Instrumento de Medição}
Dispositivo que converte uma grandeza em uma informação legível. Seus componentes principais são:
\begin{itemize}
  \item \textbf{Sensor:} capta a grandeza de interesse.
  \item \textbf{Transdutor:} converte a grandeza em outra forma (geralmente elétrica).
  \item \textbf{Indicador:} apresenta o valor final medido.
\end{itemize}

\section{Classificação dos Instrumentos}
\subsection{Por grandeza medida}
Amperímetro (corrente), voltímetro (tensão), wattímetro (potência), ohmímetro (resistência), capacímetro (capacitância), frequencímetro (frequência).

\subsection{Por tipo de indicação}
\begin{itemize}
  \item \textbf{Analógicos:} ponteiro sobre escala.
  \item \textbf{Digitais:} leitura direta em display.
\end{itemize}

\subsection{Por capacidade de armazenamento}
\begin{itemize}
  \item \textbf{Indicadores:} leitura instantânea.
  \item \textbf{Registradores:} armazenam leituras ao longo do tempo.
  \item \textbf{Totalizadores:} acumulam valores (ex: medidor de kWh).
\end{itemize}

\subsection{Por finalidade}
\begin{itemize}
  \item \textbf{Laboratoriais:} alta exatidão e precisão.
  \item \textbf{Industriais:} robustos e confiáveis, mesmo sob condições adversas.
\end{itemize}
% --- continua do arquivo anterior -------------------------
\section{Instrumentos Analógicos}

\subsection{Princípio de Funcionamento}
Os indicadores de ponteiro utilizam um \emph{galvanômetro} — essencialmente um mili‑ ou micro‑amperímetro — combinado a circuitos de condicionamento para medir diferentes grandezas. Os principais arranjos construtivos são:

\begin{description}
  \item[Bobina fixa\,/\,ferro móvel:] o campo magnético da bobina atrai uma peça ferromagnética móvel, deslocando o ponteiro.
  \item[Bobina móvel:] a bobina é montada no eixo; a interação entre seu campo e o de um ímã permanente produz o torque de deflexão.
\end{description}

\subsection{Escalas e Fundo de Escala}
A escala determina o valor mínimo e máximo que pode ser lido. Cuidados principais:

\begin{itemize}
  \item \textbf{Fundo de escala (F.S.):} valor máximo permitido; excedê‑lo pode danificar o instrumento.
  \item \textbf{Posição do zero:} à direita, central, deslocado ou suprimido.
  \item \textbf{Linearidade:} escalas podem ser homogêneas (lineares) ou heterogêneas (não lineares).
  \item \textbf{Paralaxe:} evite‑a observando o ponteiro exatamente sobre sua imagem refletida no espelho da escala.
\end{itemize}

\subsection{Ajuste de Zero em Ohmímetros}
Antes de medir resistência, curto‑circuitam‑se as pontas de prova e ajusta‑se o potenciômetro de \emph{ZERO OHM} até que o ponteiro coincida com a marca ``0 $\Omega$''.

\section{Instrumentos Digitais}

\subsection{Arquitetura Básica}
Compostos por:
\begin{enumerate}
  \item Circuito de condicionamento (converte a grandeza em tensão contínua proporcional).
  \item Conversor Analógico–Digital (ADC), muitas vezes do tipo aproximação sucessiva, flash ou $\Delta\Sigma$.
  \item Microcontrolador ou lógica de processamento.
  \item Display (\textbf{LED} ou \textbf{LCD}).
\end{enumerate}

\subsection{Displays LED vs.\ LCD}
\begin{itemize}
  \item \textbf{LED:} leitura em qualquer ângulo, visível à distância, tolera baixa luminosidade; maior consumo.
  \item \textbf{LCD:} baixo consumo, excelente sob luz solar; necessita \emph{back‑light} em pouca luz; resposta pior sob baixas temperaturas.
\end{itemize}

\subsection{Resolução e Dígitos}
A resolução relaciona‑se ao número de dígitos \emph{inteiros} mais um dígito ``meio'' (0 ou 1):
\begin{center}
\begin{tabular}{lcc}
\toprule
\textbf{Instrumento} & \textbf{Contagem Máx.} & \textbf{Exemplo de leitura} \\
\midrule
3½ dígitos & 1\,999 & \verb|199.9| \\
4½ dígitos & 19\,999 & \verb|199.99| \\
\bottomrule
\end{tabular}
\end{center}

\section{Exatidão (Classe) e Segurança}

\subsection{Classe de Exatidão}
Expressa o erro percentual relativo ao F.S.  
Ex.: classe 0,5 em um amperímetro de 200 mA implica erro máximo de $\pm1$ mA.

\subsection{Categoria de Sobretensão (CAT)}
Norma IEC 61010‑1:
\begin{description}
  \item[CAT I:] circuitos de baixa energia (eletrônica interna).  
  \item[CAT II:] cargas conectadas à tomada (eletrodomésticos).  
  \item[CAT III:] instalações fixas (quadros, motores).  
  \item[CAT IV:] origem da instalação (medidor, rede pública).  
\end{description}

Escolha sempre o instrumento com CAT e tensão de isolação adequados à aplicação.

\section{True\,RMS vs.\ Medição de Pico}
Multímetros simples calculam o valor eficaz assumindo forma senoidal pura (capturam o valor de pico). Instrumentos \textbf{True RMS} integram a potência instantânea, fornecendo o valor real eficaz mesmo para formas de onda distorcidas — essenciais em sistemas com harmônicas ou eletrônica de potência.

\section{Resumo}
Neste capítulo você aprendeu:
\begin{itemize}
  \item Conceitos fundamentais de medição, erros, exatidão e precisão.
  \item Estrutura e funcionamento de instrumentos analógicos e digitais.
  \item Importância das escalas, fundo de escala, resolução e categorias de segurança.
  \item Diferença entre medições convencionais de pico e instrumentos True RMS.
\end{itemize}

\section{Exercícios Propostos}
\begin{enumerate}
  \item Classifique os erros a seguir como grosseiros, sistemáticos ou aleatórios:  
        (a) leitura com paralaxe; (b) variação de temperatura; (c) ruído de rede elétrica.
  \item Um voltímetro de classe 1,5 e F.S.\ de 300 V indica 120 V. Determine o intervalo de incerteza.
  \item Explique por que um multímetro de 3½ dígitos não é adequado para medir pequenas variações de 0,1 mV.
  \item Indique a categoria CAT mínima para um medidor que será utilizado diretamente na entrada do quadro de força de um prédio.
\end{enumerate}
